%Copyright 2014 Jean-Philippe Eisenbarth
%This program is free software: you can 
%redistribute it and/or modify it under the terms of the GNU General Public 
%License as published by the Free Software Foundation, either version 3 of the 
%License, or (at your option) any later version.
%This program is distributed in the hope that it will be useful,but WITHOUT ANY 
%WARRANTY; without even the implied warranty of MERCHANTABILITY or FITNESS FOR A 
%PARTICULAR PURPOSE. See the GNU General Public License for more details.
%You should have received a copy of the GNU General Public License along with 
%this program.  If not, see <http://www.gnu.org/licenses/>.

%Based on the code of Yiannis Lazarides
%http://tex.stackexchange.com/questions/42602/software-requirements-specification-with-latex
%http://tex.stackexchange.com/users/963/yiannis-lazarides
%Also based on the template of Karl E. Wiegers
%http://www.se.rit.edu/~emad/teaching/slides/srs_template_sep14.pdf
%http://karlwiegers.com
\documentclass[16pt]{scrreprt}
\usepackage{listings}
\usepackage{float}
\usepackage{setspace}
\usepackage{url}
\usepackage{longtable}
\usepackage{booktabs}
\usepackage{underscore}
\usepackage{graphicx, subfig}
\usepackage[bookmarks=true]{hyperref}
\usepackage[utf8]{inputenc}
\usepackage[english]{babel}

\hypersetup{
    bookmarks=false,    % show bookmarks bar?
    pdftitle={Software Requirement Specification},    % title
    pdfauthor={Jean-Philippe Eisenbarth},                     % author
    pdfsubject={TeX and LaTeX},                        % subject of the document
    pdfkeywords={TeX, LaTeX, graphics, images}, % list of keywords
    colorlinks=true,       % false: boxed links; true: colored links
    linkcolor=blue,       % color of internal links
    citecolor=black,       % color of links to bibliography
    filecolor=black,        % color of file links
    urlcolor=purple,        % color of external links
    linktoc=page            % only page is linked
}%
\def\myversion{1.0 }
\date{}
%\title
\usepackage{hyperref}
\DeclareOldFontCommand{\bf}{\normalfont\bfseries}{\mathbf}
\begin{document}

\begin{flushright}
    \rule{16cm}{5pt}\vskip1cm
    \begin{bfseries}
        \LARGE{SOFTWARE REQUIREMENTS\\ SPECIFICATION}\\
        \vspace{0.8cm}
        for\\
        \vspace{0.8cm}
        iSport\\
        \begin{figure}[h]
		\flushright
  		\includegraphics[width=.2\textwidth]{logo.png}
		\end{figure}
        \vspace{0.8cm}
        \normalsize{Version \myversion approved}\\
        \vspace{1.0cm}
        Prepared by\vspace{0.5cm} \\ 
        \normalsize{1650940 Jiang Xiaohu\\1650932 Xu Jingnan\\1651058 Wang Yicheng}
        \\
        \vspace{1.1cm}
        \normalsize{School of Software Engineering\\ Tongji University}\\
        \vspace{1.0cm}
        \today\\
    \end{bfseries}
\end{flushright}

\tableofcontents


\chapter*{Revision History}

\begin{center}
    \begin{tabular}{|c|c|c|c|}
        \hline
	    Name & Date & Reason For Changes & Version\\
        \hline
	    Jiang Xiaohu & 2019.11.1 & Finish Introduction Part  & v1.0\\
        \hline
	    31 & 32 & 33 & 34\\
        \hline
    \end{tabular}
\end{center}

\chapter{Introduction}

\section{SRS Purpose}
The purpose of this document is to present a detailed description of iSport. It will explain the purpose and features of iSport, the interfaces of the iSport, the functional and nonfunctional requirements of iSport, what iSport will do, and the constraints under which it must operate and how the system will react to external stimuli. This document is intended for both the stakeholders and the developers of the system and will be proposed to the clients for its approval.
%This SRS (Software Requirements Specification) document aims to identify the product--iSport whose software requirements are specified in this 
%document, including the revision or release number. And the scope of iSport is covered by this SRS, too, particularly if this SRS describes only 
%part of the system or a single subsystem.$>$

\section{Product Scope}
This software system will be a web based system for sports fans, professional athletes, and patients who are under recovering training. \\
\\
This system will be designed to maximize the exercising efficiency by providing tools to assist in checking and correcting user's wrong postures and recommending training courses customized for users, which would otherwise have to be expensive, time-consuming and labor intensive. By maximizing the user’s training efficiency and convenience the system will meet the needs of sports fans, athletes and injured patients while remaining easy to understand and use.\\
\\
More specifically, this system is designed to allow a user to imitate the standard exercising postures while observe and correct their mistakes simultaneously with the help of a website. \\
\\
The software will collect some professional courses in the database, including static and dynamic trainings which means doing exercise according to a set of images or a video and iSport will recommend suitable trainings for users on the basis of their training performance. Courses are classified into exercising courses and recovering courses, aiming to help athletes and patients respectively.\\
\\
Both visual and audio notification are used in every course of the system to provide eye-catching, user-friendly and clear instructions; the feedback of one's training is proposed once the training is over and the report can be browsed in the report page.\\
\\
The selection and deletion of one user's favorable course is supported in personal information webpage and one can comment training he/she has taken on comment webpage to provide suggestions to other users.\\
\\
 The personal information registering, changes is allowed via the application options. The system also contains a relational database containing a list of users, training images and videos.

\section{Intended Audience}
This Software Requirements document is intended for:
\begin{itemize}
	\item Developers who can review project’s capabilities and more easily understand where their efforts should be targeted to improve or add more features to it (design and code the application – it sets the guidelines for future development).
	\item Project testers can use this document as a base for their testing strategy as some bugs are easier to find using a requirements document. This way testing becomes more methodically organized.
   \item End users of this application who wish to read about what this project can do.
   \item Clients who delegate the software development to our team, and can check if their requirements are perfectly understood by the developers and whether the functional and nonfunctional requirements are entirely meet. And they can modify some requirements according to this document in later stage. 
   \item Project managers who translate the clients' requirement to the programmers and supervise them to implement the requirements mentioned in the documents and if the clients modify the requirements, the project managers have to negotiate the changes with the clients and modify the document.
   \item Marketing staff who are responsible to sale and prompt the software should be clear about what iSport can do, what iSport's advantages are, what iSport's competitive power is.
   \item Service staff who are responsible to solve the customer's problems, they have to know what iSport can do and feedback the technical problems to developers.
   \item Document writers who are responsible for writing the rest documents in later development stages should follow the requirements specified in this SRS.
\end{itemize}
The rest part of this SRS document contains the overall description of iSport, and iSport's specific, nonfunctional and other requirements which are shown in chapter\ref{Overall Description}, chapter\ref{Specific Requirements}, chapter\ref{Other Nonfunctional Requirements} and chapter\ref{Other Requirements} respectively.
We suggest the readers begin with the overview sections and proceeding through the sections that are most pertinent to them:
\begin{itemize}
	\item Developers and project testers are recommended to focus on specific and nonfunctional requirements part because theses parts will lead them to build qualified, safe and satisfying application and theses parts are all related to coding (construction and verification stage).
	\item Clients, project managers and document writers should focus on the entire document since they are responsible for all the requirements specified in this paper.
	\item Marketing staff have to focus on the functional part of iSport.
\end{itemize}


\section{Definitions, Acronyms, and Abbreviations}
\begin{longtable}{|p{1.9in}|p{4in}|c|}
xxxxx & xxxxxx  \kill
\caption{Definitions\label{simple}}\\ \hline
\multicolumn{3}{|c|}{\bf Definitions, Acronyms, and Abbreviations}\\ \hline
\endfirsthead
\caption[]{(continued)}\\ \hline
\multicolumn{3}{|c|}{\bf Definitions, Acronyms, and Abbreviations (continued)}\\
\hline
\endhead
\hline
\multicolumn{3}{|c|}{\bf Continued $\ldots$}\\
\hline
\endfoot
\hline
\multicolumn{3}{|c|}{\bf The End}\\
\hline
\endlastfoot
Term & Definitions  \\
\hline
User & Someone who interacts with iSport including sports fans, athletes and  injured patients who need recovery training.\\  \hline  
Sports fans & One of iSport's potential customers who love sports and want to get professional instructions when exercising. Some of them may can't afford the expense of personal coaching or don't have time to go to the gym. \\ \hline
Athletes & One of iSport's potential customers who want to get real-time exercising feedback to improve their performance or who want to get some relaxing training in their spare time to keep a good competitive state.\\  \hline
Injured Patients & One of iSport's potential customers who need recovering training after some treatments, e.g. surgeries. On the one hand, some of them may can't afford the doctor's expensive medical instructions for recovering training. on the other hand, there is no enough doctors or nurses who can instruct and supervise the patients' recovering exercising. But without professional instructions training can be useless or even leads to secondary trauma.\\  \hline
Admin/Administrator & System administrator who is given specific permission for managing and controlling the system, e.g. updating the user's information, uploading new training courses.\\ \hline
User Info & User's basic information including user's avatar, account name, tel-number and email address.\\ \hline
Courses & Training courses including normal exercising training and recovering training.\\ \hline
Normal Courses & Training courses which serve the sports fans and athletes.\\ \hline
Recovering Courses & Training courses which serve the injured patients.\\ \hline
Static Courses & Training courses which instruct the users photo by photo.\\ \hline
Dynamic Courses & Training courses which instruct the users according to a standard video.\\ \hline
Appraisal Subsystem & Remark the user's performance by using a grade from 0 -100\\ \hline
Comment Subsystem & User comment on the training courses they have taken to provide reference for other users.\\ 
\hline 
Recommendation Subsystem & A subsystem which will provide some courses for users according to their recent performance.\\ 
\hline
Exercise Tips & There will be sports tips in the webpage of iSport to prevent users from athletic injuries.\\ 
\hline 
Sport Report & A web page to feedback the user's exercising performance.\\ 
\hline 
Audio Notification & An audio notification will be shown when the user is doing exercise to encourage the user to hold on or notify the user to correct their postures.\\ 
\hline 
Visual Notification & A visual notification will be shown when the user is doing exercise, if the user's posture is standard, then the web-frame will turn green to suggest the user to hold on, otherwise the web-frame will be red.\\ 
\hline 
DataBase & A relational database containing a list of user info, training images and videos.\\ 
\hline 
Detection Subsystem & Subsystem to detect the user's postures and draw the user's skeleton. The main model of detection subsystem is PoseNet.\\ 
\hline 
Comparison Subsystem & Subsystem to compare the postures of the user and that of the standard. The subsystem aims to check if the user pass the posture.\\ 
\hline 
Correction Subsystem & Subsystem to calculate where the postures' wrong part are, e.g. left-arm, right-leg, head.\\ 
\hline 
Clients & Group who delegate the development of iSport to the developers and will take charge of the later management of iSport.\\ \hline
Developers & Develop team including project managers, programmers, testers who are responsible for the development of iSport and its later mainteinance and updating.\\ 
\hline 
\end{longtable}

\section{Document Conventions}
This document follows MLA Format[1]. Bold-faced text has been used to emphasize sectionand sub-section headings. Highlighting is to point out the references of tables and figures. And italicized text is used to label and recognize special characters or terminologies\\

The SRS paper is written by latex[4], the packages we used to format the document including \small{$longtable, graphicx, subfig, utf8-inputenc,hyperref$ }and so on.\\
And we modified the IEEE SRS latex template from Jean-Philippe Eisenbarth's github[5] according to the IEEE standard[2] mentioned in Frank F Tusi'book[3].


\section{References and Acknowledgements}
\subsubsection{Standard Reference}
The standard we have followed are as follows:\\

$[1]$ T. Russell, A. Brizee, E. Angeli, and R. Keck, “Mla formatting and style guide,” The Purdue OWL, 2010.\\

$[2]$ I. S. E. S. Committee et al., “Ieee recommended practice for software re- quirements specifications,” IEEE organization, 1998.\\


$[3]$ F. F. Tsui, O. Karam, and B. Bernal, Essentials of software engineering. Jones Bartlett Learning, 2016.

\subsubsection{Writing Tools Reference}
The writing tools we have used are as follows:\\

$[4]$L. Lamport, LATEX: a document preparation system: user’s guide and reference manual. Addison-wesley, 1994.\\

$[5]$ J.-P. Eisenbarth, “Srs latex template under ieee standard,” \\http://https://github.com/jpeisenbarth/SRS-Tex.



\chapter{Overall Description}
\label{Overall Description}

\section{Product Perspective}
$<$Describe the context and origin of the product being specified in this SRS.  
For example, state whether this product is a follow-on member of a product 
family, a replacement for certain existing systems, or a new, self-contained 
product. If the SRS defines a component of a larger system, relate the 
requirements of the larger system to the functionality of this software and 
identify interfaces between the two. A simple diagram that shows the major 
components of the overall system, subsystem interconnections, and external 
interfaces can be helpful.$>$

\section{Product Functionality}
$<$Summarize the major functions the product must perform or must let the user 
perform. Details will be provided in Section 3, so only a high level summary 
(such as a bullet list) is needed here. Organize the functions to make them 
understandable to any reader of the SRS. A picture of the major groups of 
related requirements and how they relate, such as a top level data flow diagram 
or object class diagram, is often effective.$>$

\section{Users and Characteristics}
$<$Identify the various user classes that you anticipate will use this product.  
User classes may be differentiated based on frequency of use, subset of product 
functions used, technical expertise, security or privilege levels, educational 
level, or experience. Describe the pertinent characteristics of each user class.  
Certain requirements may pertain only to certain user classes. Distinguish the 
most important user classes for this product from those who are less important 
to satisfy.$>$

\section{Operating Environment}
$<$Describe the environment in which the software will operate, including the 
hardware platform, operating system and versions, and any other software 
components or applications with which it must peacefully coexist.$>$

\section{Design and Implementation Constraints}
$<$Describe any items or issues that will limit the options available to the 
developers. These might include: corporate or regulatory policies; hardware 
limitations (timing requirements, memory requirements); interfaces to other 
applications; specific technologies, tools, and databases to be used; parallel 
operations; language requirements; communications protocols; security 
considerations; design conventions or programming standards (for example, if the 
customer’s organization will be responsible for maintaining the delivered 
software).$>$

\section{User Documentation}
$<$List the user documentation components (such as user manuals, on-line help, 
and tutorials) that will be delivered along with the software. Identify any 
known user documentation delivery formats or standards.$>$

\section{Assumptions and Dependencies}

$<$List any assumed factors (as opposed to known facts) that could affect the 
requirements stated in the SRS. These could include third-party or commercial 
components that you plan to use, issues around the development or operating 
environment, or constraints. The project could be affected if these assumptions 
are incorrect, are not shared, or change. Also identify any dependencies the 
project has on external factors, such as software components that you intend to 
reuse from another project, unless they are already documented elsewhere (for 
example, in the vision and scope document or the project plan).$>$


\chapter{Specific Requirements}
\label{Specific Requirements}
\section{External Interfaces Requirements}
$<$ $>$
%\section{User Interfaces}
%$<$Describe the logical characteristics of each interface between the software 
%product and the users. This may include sample screen images, any GUI standards 
%or product family style guides that are to be followed, screen layout 
%constraints, standard buttons and functions (e.g., help) that will appear on 
%every screen, keyboard shortcuts, error message display standards, and so on.  
%Define the software components for which a user interface is needed. Details of 
%the user interface design should be documented in a separate user interface 
%specification.$>$

\section{Functional Requirements}
$<$ $>$
%\section{Hardware Interfaces}
%$<$Describe the logical and physical characteristics of each interface between 
%the software product and the hardware components of the system. This may include 
%the supported device types, the nature of the data and control interactions 
%between the software and the hardware, and communication protocols to be 
%used.$>$

\section{Behavior Requirements}
$<$ $>$

%\section{Software Interfaces}
%$<$Describe the connections between this product and other specific software 
%components (name and version), including databases, operating systems, tools, 
%libraries, and integrated commercial components. Identify the data items or 
%messages coming into the system and going out and describe the purpose of each.  
%Describe the services needed and the nature of communications. Refer to 
%documents that describe detailed application programming interface protocols.  
%Identify data that will be shared across software components. If the data 
%sharing mechanism must be implemented in a specific way (for example, use of a 
%global data area in a multitasking operating system), specify this as an 
%implementation constraint.$>$

%\section{Communications Interfaces}
%$<$Describe the requirements associated with any communications functions 
%required by this product, including e-mail, web browser, network server 
%communications protocols, electronic forms, and so on. Define any pertinent 
%message formatting. Identify any communication standards that will be used, such 
%as FTP or HTTP. Specify any communication security or encryption issues, data 
%transfer rates, and synchronization mechanisms.$>$


%\chapter{System Features}
%$<$This template illustrates organizing the functional requirements for the 
%product by system features, the major services provided by the product. You may 
%prefer to organize this section by use case, mode of operation, user class, 
%object class, functional hierarchy, or combinations of these, whatever makes the 
%most logical sense for your product.$>$
%
%\section{System Feature 1}
%$<$Don’t really say “System Feature 1.” State the feature name in just a few 
%words.$>$
%
%\subsection{Description and Priority}
%$<$Provide a short description of the feature and indicate whether it is of 
%High, Medium, or Low priority. You could also include specific priority 
%component ratings, such as benefit, penalty, cost, and risk (each rated on a 
%relative scale from a low of 1 to a high of 9).$>$
%
%\subsection{Stimulus/Response Sequences}
%$<$List the sequences of user actions and system responses that stimulate the 
%behavior defined for this feature. These will correspond to the dialog elements 
%associated with use cases.$>$
%
%\subsection{Functional Requirements}
%$<$Itemize the detailed functional requirements associated with this feature.  
%These are the software capabilities that must be present in order for the user 
%to carry out the services provided by the feature, or to execute the use case.  
%Include how the product should respond to anticipated error conditions or 
%invalid inputs. Requirements should be concise, complete, unambiguous, 
%verifiable, and necessary. Use “TBD” as a placeholder to indicate when necessary 
%information is not yet available.$>$
%
%$<$Each requirement should be uniquely identified with a sequence number or a 
%meaningful tag of some kind.$>$
%
%REQ-1:	REQ-2:
%
%\section{System Feature 2 (and so on)}


\chapter{Other Nonfunctional Requirements}
\label{Other Nonfunctional Requirements}
\section{Performance Requirements}
$<$If there are performance requirements for the product under various 
circumstances, state them here and explain their rationale, to help the 
developers understand the intent and make suitable design choices. Specify the 
timing relationships for real time systems. Make such requirements as specific 
as possible. You may need to state performance requirements for individual 
functional requirements or features.$>$

\section{Safety and Security Requirements}
$<$Specify those requirements that are concerned with possible loss, damage, or 
harm that could result from the use of the product. Define any safeguards or 
actions that must be taken, as well as actions that must be prevented. Refer to 
any external policies or regulations that state safety issues that affect the 
product’s design or use. Define any safety certifications that must be 
satisfied.$>$
%\section{Security Requirements}
$<$Specify any requirements regarding security or privacy issues surrounding use 
of the product or protection of the data used or created by the product. Define 
any user identity authentication requirements. Refer to any external policies or 
regulations containing security issues that affect the product. Define any 
security or privacy certifications that must be satisfied.$>$

\section{Software Quality Attributes}
$<$Specify any additional quality characteristics for the product that will be 
important to either the customers or the developers. Some to consider are: 
adaptability, availability, correctness, flexibility, interoperability, 
maintainability, portability, reliability, reusability, robustness, testability, 
and usability. Write these to be specific, quantitative, and verifiable when 
possible. At the least, clarify the relative preferences for various attributes, 
such as ease of use over ease of learning.$>$

%\section{Business Rules}
%$<$List any operating principles about the product, such as which individuals or 
%roles can perform which functions under specific circumstances. These are not 
%functional requirements in themselves, but they may imply certain functional 
%requirements to enforce the rules.$>$


\chapter{Other Requirements}
\label{Other Requirements}
$<$Define any other requirements not covered elsewhere in the SRS. This might 
include database requirements, internationalization requirements, legal 
requirements, reuse objectives for the project, and so on. Add any new sections 
that are pertinent to the project.$>$

\section{Data Dictionary Requirements}
$<$ $>$

%\section{Appendix A: Glossary}
%%see https://en.wikibooks.org/wiki/LaTeX/Glossary
%$<$Define all the terms necessary to properly interpret the SRS, including 
%acronyms and abbreviations. You may wish to build a separate glossary that spans 
%multiple projects or the entire organization, and just include terms specific to 
%a single project in each SRS.$>$

\section{Appendix A: Analysis Models}
$<$Optionally, include any pertinent analysis models, such as data flow 
diagrams, class diagrams, state-transition diagrams, or entity-relationship 
diagrams.$>$

\section{Appendix B: To Be Determined List}
$<$Collect a numbered list of the TBD (to be determined) references that remain 
in the SRS so they can be tracked to closure.$>$

\end{document}
